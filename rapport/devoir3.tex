\documentclass[12pt,a4paper]{article}

\usepackage{fullpage}
\usepackage[utf8]{inputenc}
\usepackage{amsfonts}
\usepackage{amssymb}
\usepackage{listings}
\usepackage[french]{babel}
\usepackage[cyr]{aeguill}

\def\dollar{\$}

\setlength{\parindent}{2em}
\setlength{\parskip}{1em}

\newcommand{\quotes}[1]{``#1''}

\begin{document}

\begin{titlepage}
\centering
{\scshape\LARGE Université de Bordeaux \par}
{\scshape\Large Master 1 Informatique \par}
\vspace{3cm}

{\Huge\bfseries Devoir Système d'Exploitation \par}%%%%
\vspace{0.5cm}
{\Large\itshape Rapport 3: NACHOS Pagination \par}

\vfill
réalisé par \par
Quoc-Nghia \textsc{NGUYEN} \par
Aghiles \textsc{CHAOUCHI} \par
Nadhir \textsc{HOUARI} \par
\vfill

{\large 19 décembre 2016\par}

\end{titlepage}

\section{Bilan}
Le but de ce troisième devoir est de mettre en place un mécanisme de gestion de mémoire paginée simple au sein du système. Notre réalisation se divise en deux parties:

\begin{enumerate}
\item Créer une classe \textbf{PageProvider} qui encapsule les mécanismes de gestions des pages de mémoire physiques en profitant d'un tableau de type \textbf{BitMap}. \textbf{PageProvider} doit permettre de: (1) récupérer le numéro d'une page libre et initialisé à 0 (méthode \textit{GetEmptyPage}); (2) libérer une page obtenue par \textit{GetEmptyPage} (méthode \textit{ReleasePage}); (3) demander combien de pages restent disponibles ? (méthode \textit{NumAvailPage}).
\item Implémenter l'appel système \textbf{ForkExec} qui permet de lancer un programme utilisateur dans un nouveau processus et assurer que ces processus utilisateurs fonctionnent correctement en terme d'allocation de mémoire et peuvent contenir eux-même des threads lancés avec l'appel système \textbf{ThreadCreate} qu'on a implémenté lors du dernier TP.
\end{enumerate}

Notre réalisation semble fonctionner correctement car elle passe nos jeux de test et rend bien les résultats attendus. Cependant cette fois-ci en exécutant explicitement \textbf{valgrind --leak-check=full} pour chaque test, on constate une fuite de mémoire qui aurait son origine dans la création des processus utilisateurs. Ce problème sera expliqué ultérieurement dans ce rapport.

En outre, nous n'arrivons pas à finir deux derniers bonus (Action II.7 et Partie III).

\section{Points délicats}

Afin de pouvoir lancer plusieurs processus utilisateurs dont chacun contient plusieurs threads utilisateurs, on a augmenté les limites pour le mémoire comme suit:
\begin{itemize}
\item NumPhysPages = 1024
\item UserStackArea = 1024x4
\item NB\_MAX\_THREAD = 4x4
\end{itemize}

\subsection{PageProvider et Stratégie d'allocation des pages}
Comme chaque machine n'a besoin d'un seul PageProvider, on a décidé d'initialiser ce dernier au niveau du système, dans la méthode \textbf{Initialize()} du fichier \textit{threads/system.cc} qui est en charge d'initialiser toutes les structures de données globales de Nachos. En plus, on a mis en place un compteur \textbf{number\_proc} qui permet de savoir il y a combien de processus qui sont lancés, afin de par exemple décider quand on peut terminer le thread noyau.

La stratégie d'allocation des pages qu'on a choisi est basée sur un tableau de type BitMap dont chaque créneau correspond à une page. Tant qu'il y a des créneaux libres, autrement dit il y a des pages libres, on peut l'allouer aux demandeurs. Sinon la demande sera verrouillée par un sémaphore dont le niveau initial est \textbf{NumPhysPages} jusqu'à ce qu'une place soit disponible.

\subsection{Terminaison d'un processus et celle d'un thread}
Par convention, un processus se termine si et seulement si tous ses threads sont déjà disparus, et un programme se termine une fois que le nombre des processus lancés baisse à 0. Ainsi si jamais un processus veut s'arrêter, on se contentera donc de terminer forcément tous ses threads.

Grâce au tableau \textbf{buffThread} dans \textit{AddrSpace} qui stocke tous les threads qu'on a créé, on peut faire en sorte que quand un processus s'arrête, tous ses threads utilisateurs doivent s'arrêter également en appelant \textbf{Finish()} sur chacun des threads.

De façon similaire, on marque les threads stockés par un indicateur booléen (true ou false) pour savoir si un thread est en train d'utiliser \textbf{PutString}. Au bout d'un moment où un thread est terminé de façon brutale, on va appeler \textbf{V()} sur la sémaphore de PutString autant de fois que le nombre des processus qui n'a pas encore fini avec PutString pour enlever tous les verrous qu'on a mis.

\section{Limitations}
Il y a une fuite de mémoire que nous n'arrivons pas à résoudre. Cela est à l'origine dans notre implémentation de l'appel système \textbf{ForkExec} (\textit{userprog/userfork.cc:47}, l'instanciation de l'espace d'adressage du nouveau processus utilisateur). Tout essai de libérer \textit{*space} rend à un "segmentation fault".

\section{Tests}
Vous trouverez donc dans le répertoire \verb$./test$ les tests \verb$tp3_*$ qui peuvent être lancés directement depuis le répertoire \verb$code$ avec la commande \texttt{./userprog/nachos -rs 1234 -x /test/tp3\_nom\_du\_test}
\end{document}

